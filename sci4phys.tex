%
\documentclass[
  notoc % Suppress Tufte style table of contents.
]{tufte-book}

% Required Tufte packages.
\usepackage{changepage} % or changepage
\usepackage{fancyhdr}
\usepackage{fontenc}
\usepackage{geometry}
\usepackage{hyperref}
\usepackage{natbib}
\usepackage{bibentry}
\usepackage{optparams}
\usepackage{paralist}
\usepackage{placeins}
\usepackage{ragged2e}
\usepackage{setspace}
\usepackage{textcase}
\usepackage{textcomp}
\usepackage{titlesec}
\usepackage{titletoc}
\usepackage{xcolor}
\usepackage{xifthen}
\usepackage{morefloats}

\geometry{paperheight=10in,paperwidth=7in,marginparwidth=30mm,marginparsep=2mm,bindingoffset=10mm,top=10mm,inner=8mm,outer=8mm,bottom=16mm,includehead,includemp}

% Tufte vs. Pandoc workaround.
% Issue: https://github.com/Tufte-LaTeX/tufte-latex/issues/64.
\renewcommand\allcapsspacing[1]{{\addfontfeature{LetterSpace=15}#1}}
\renewcommand\smallcapsspacing[1]{{\addfontfeature{LetterSpace=10}#1}}

% \setmainfont{TeX Gyre Pagella}
\usepackage[utf8]{inputenc}
\usepackage[T1]{fontenc}
\setmainfont{texgyrepagella}[
  Extension = .otf,
  UprightFont = *-regular,
  BoldFont = *-bold,
  ItalicFont = *-italic,
  BoldItalicFont = *-bolditalic,
]

\usepackage{fontspec}
\setmonofont{JuliaMono-Medium.ttf}[
    % Do not remove the trailing forward slash.
    Path = /home/runner/.julia/artifacts/45f34ceb7f1b7b67949715de56b123afeaa72e47/juliamono-0.045/,
    Contextuals = Alternate,
    Ligatures = NoCommon
]

\newfontface\JuliaMonoRegular{JuliaMono-Regular.ttf}[
    Path = /home/runner/.julia/artifacts/45f34ceb7f1b7b67949715de56b123afeaa72e47/juliamono-0.045/,
    Contextuals = Alternate,
    Ligatures = NoCommon
]

\newfontface\JuliaMonoBold{JuliaMono-Bold.ttf}[
    Path = /home/runner/.julia/artifacts/45f34ceb7f1b7b67949715de56b123afeaa72e47/juliamono-0.045/,
    Contextuals = Alternate,
    Ligatures = NoCommon
]



\usepackage{graphicx}
\makeatletter
\def\maxwidth{\ifdim\Gin@nat@width>\linewidth\linewidth\else\Gin@nat@width\fi}
\def\maxheight{\ifdim\Gin@nat@height>\textheight\textheight\else\Gin@nat@height\fi}
\makeatother
% Scale images if necessary, so that they will not overflow the page
% margins by default, and it is still possible to overwrite the defaults
% using explicit options in \includegraphics[width, height, ...]{}
\setkeys{Gin}{width=\maxwidth,height=\maxheight,keepaspectratio}
\DeclareRobustCommand{\href}[2]{#2\footnote{\url{#1}}}

\usepackage{float}
\floatplacement{figure}{H}

% Listings Julia syntax definition.
\input{/home/runner/.julia/packages/Books/odNoe/defaults/julia_listings.tex}

% Unicode support.
\input{/home/runner/.julia/packages/Books/odNoe/defaults/julia_listings_unicode.tex}

% Used by Pandoc.
\providecommand{\tightlist}{%
  \setlength{\itemsep}{0pt}\setlength{\parskip}{0pt}
}
\newcommand{\passthrough}[1]{#1}

\usepackage{longtable}
\usepackage{booktabs}
\usepackage{array}

% Source: Wandmalfarbe/pandoc-latex-template.

\definecolor{linkblue}{HTML}{117af2}
\usepackage{hyperref}
\hypersetup{
  colorlinks,
  citecolor=linkblue,
  linkcolor=linkblue,
  urlcolor=linkblue,
  linktoc=page, % Avoid Table of Contents being nearly completely blue.
  pdftitle={Scientific Computing for Physicist},
  pdfauthor={Yu-Sheng Zhao; Yi-Dai Zhang; Jin-Guo Liu},
  pdflang={en-US},
  breaklinks=true,
  pdfcreator={LaTeX via Pandoc}%
}
\urlstyle{same} % disable monospaced font for URLs

\title{Scientific Computing for Physicist}
\author{\noindent{Yu-Sheng Zhao}\\[3mm] \noindent{Yi-Dai
Zhang}\\[3mm] \noindent{Jin-Guo Liu}\\[3mm] }
\date{}

% Re-enable section numbering which was disabled by tufte.
\setcounter{secnumdepth}{2}

% Fix captions for longtable.
% Thanks to David Carlisle at https://tex.stackexchange.com/a/183344/92217.
\makeatletter
\def\LT@makecaption#1#2#3{%
  \noalign{\smash{\hbox{\kern\textwidth\rlap{\kern\marginparsep
  \parbox[t]{\marginparwidth}{\vspace{12pt}%
\@tufte@caption@font \@tufte@caption@justification \noindent
   #1{#2: }\ignorespaces #3}}}}}}
\makeatother

% Doesn't seem to do anything.
\usepackage{float}
\floatplacement{figure}{H}
\floatplacement{table}{H}

% Reduce large spacing around sections.
\titlespacing*{\chapter}{0pt}{5pt}{20pt}
\titlespacing*{\section}{0pt}{2.5ex plus 1ex minus .2ex}{1.3ex plus .2ex}
\titlespacing*{\subsection}{0pt}{1.75ex plus 1ex minus .2ex}{1.0ex plus.2ex}

\titleformat{\chapter}%
  [hang]% shape
  {\normalfont\huge\itshape}% format applied to label+text
  {\huge\thechapter}% label
  {1em}% horizontal separation between label and title body
  {}% before the title body
  []% after the title body

% Reduce spacing in table of contents.
\usepackage{etoolbox}
\makeatletter
\pretocmd{\chapter}{\addtocontents{toc}{\protect\addvspace{-3\p@}}}{}{}
\pretocmd{\section}{\addtocontents{toc}{\protect\addvspace{-4\p@}}}{}{}
\pretocmd{\subsection}{\addtocontents{toc}{\protect\addvspace{-5\p@}}}{}{}
\makeatother

% Long texts are harder to read than tables.
% Therefore, we can reduce the font size of the table.
\AtBeginEnvironment{longtable}{\footnotesize}

% Some space between paragraphs is necessary because code blocks can output single line paragraphs.
\setlength\parskip{1em plus 0.1em minus 0.2em}

% For justified text.
\usepackage{ragged2e}

% tufte-book disables subsubsections by default.
% Got this definition back via `\show\subsubsection`.

\usepackage{amsfonts}
\usepackage{amssymb}
\usepackage{amsmath}
\usepackage{unicode-math}

% URL line breaks.
\usepackage{xurl}

% Probably doesn't hurt.
\usepackage{marginfix}




\begin{document}

\makeatletter
\thispagestyle{empty}
\vfill
{\Huge\bf
\noindent
\@title
}\\[1in]
{\Large
\noindent
\@author
}
\makeatother

\makeatletter
\newpage
\thispagestyle{empty}
\vfill
{\noindent
\begin{tabular}{l} Yu-Sheng Zhao\\ Hong-Kong University of Science and Technology (Guangzhou)\\ \\ Yi-Dai Zhang\\ Hong-Kong University of Science and Technology (Guangzhou)\\ \\ Jin-Guo Liu\\ Hong-Kong University of Science and Technology (Guangzhou)\\ \\ \end{tabular}
}
\vfill
{\small
\url{https://book.jinguo-group.science}

Version: 2024-02-07

Creative Commons Attribution-NonCommercial-ShareAlike 4.0 International
}
\makeatother


% Don't remove this or authors will show up in header of every page.
\frontmatter
\mainmatter
\fancyfoot[C]{\url{https://github.com/johndoe/Book.jl}}

\setcounter{tocdepth}{1}
\tableofcontents

% Justify text.
\justifying

% parindent seems to be set from within another class too.
% it is really not useful here because it will also indent lines directly after
% code blocks. Which most of the times not useful.
\setlength{\parindent}{0pt}

\hypertarget{sec:open-source-dev-toolchains}{%
\chapter{Becoming an Open-Source
Developer}\label{sec:open-source-dev-toolchains}}

This section focuses on understanding the open source workflow, which is
the foundation of scientific computing. Along the way, we will introduce
to you our recommended tools for accomplishing each task.

\hypertarget{sec:terminal}{%
\section{Get a Terminal!}\label{sec:terminal}}

You need to get a working terminal to follow the instructions in this
book, because every cool guy uses a terminal.

\hypertarget{sec:linux}{%
\subsection{Linux operating system}\label{sec:linux}}

Using Linux is the most straight-forward way to get a terminal. Just
like Windows, IOS, and macOS, Linux is an operating system. In fact,
Android, one of the most popular platforms on the planet, is powered by
the Linux operating system. It is free to use,
\href{https://opensource.com/resources/what-open-source}{open source},
widely used on clusters and good at automating your works. Linux kernel
and Linux distribution are different concepts. - The \textbf{Linux
kernel} is started
by~\href{https://en.wikipedia.org/wiki/Linus_Torvalds}{Linus Torvalds}
in 1991. - A~\textbf{Linux distribution} is
an~\href{https://en.wikipedia.org/wiki/Operating_system}{operating
system}~made from a software collection that includes
the~\href{https://en.wikipedia.org/wiki/Linux_kernel}{Linux kernel}~and,
often,
a~\href{https://en.wikipedia.org/wiki/Package_management_system}{package
management system}. The Linux distribution used in this course is
\href{https://ubuntu.com/desktop}{Ubuntu}.

\hypertarget{sec:shell}{%
\subsection{Shell (or Terminal)}\label{sec:shell}}

Although you can use a \textbf{graphical user interface} (GUI) to
interact with your Linux distribution, you will find that the
\textbf{command line interface} (CLI) is more efficient and powerful.
The CLI is also known as the \textbf{shell} or \textbf{terminal}.

The shell is a program that takes commands from the keyboard and gives
them to the operating system to perform. \href{https://zsh.org/}{Zsh}
and \href{https://gnu.org/software/bash/}{Bash} are two popular shell
interpreters used in the Linux operating systems. - \textbf{Bash} is the
default shell on most Linux distributions. - \textbf{Zsh} (with
\href{https://github.com/ohmyzsh/ohmyzsh}{oh-my-zsh} extension) is an
extended version of the shell, with a more powerful command-line editing
and completion system. It includes features like spelling correction and
tab-completion, and it also supports plugins and themes.

In Ubuntu, one can use \passthrough{\lstinline!Ctrl!} +
\passthrough{\lstinline!Alt!} + \passthrough{\lstinline!T!} to open a
shell. In a shell, we use - \passthrough{\lstinline!man command\_name!}
to get help information related to a command, -
\passthrough{\lstinline!CTRL-C!} to break a program and -
\passthrough{\lstinline!CTRL-D!} to exit a shell or an REPL.

The following is a short list of bash commands that will be used
frequently in this book.

\begin{lstlisting}
man     # an interface to the system reference manuals

ls      # list directory contents
cd      # change directory
mkdir   # make directories
rm      # remove files or directories
pwd     # print name of current/working directory

echo    # display a line of text
cat     # concatenate files and print on the standard output

alias   # create an alias for a command

lscpu   # display information about the CPU architecture
lsmem   # list the ranges of available memory with their online status

top     # display Linux processes
ssh     # the OpenSSH remote login client
vim     # Vi IMproved, a programmer's text editor
git     # the stupid content tracker

tar     # an archiving utility
\end{lstlisting}

\textbf{Git Resources}

\begin{itemize}
\tightlist
\item
  \href{https://missing.csail.mit.edu/2020/shell-tools/}{MIT Open
  course: Missing semester}
\item
  \href{https://learn.microsoft.com/en-us/training/paths/shell/}{Get
  started with the Linux command line and the Shell}
\end{itemize}

\hypertarget{sec:editor}{%
\subsection{Vim Editor}\label{sec:editor}}

To edit files in the terminal, you can use \passthrough{\lstinline!Vim!}
- the default text editor in most Linux distributions.
\passthrough{\lstinline!Vim!} has three primary modes, each tailored for
specific tasks. The primary modes include - \textbf{Normal Mode}, where
users can navigate through the file and perform tasks like deleting
lines or copying text; One can enter the normal mode by typing
\passthrough{\lstinline!ESC!}; - \textbf{Insert Mode}, where users can
insert text as in conventional text editors; One can enter the insert
mode by typing \passthrough{\lstinline!i!} in the normal mode; -
\textbf{Command Mode}, where users input commands for tasks like saving
files or searching; One can enter the command mode by typing
\passthrough{\lstinline!:!} in the normal mode.

A few commands are listed below to get you started with
\passthrough{\lstinline!Vim!}.

\begin{lstlisting}
i       # input
:w      # write
:q      # quit
:q!     # force quit without saving

u       # undo
CTRL-R  # redo
\end{lstlisting}

All the commands must be executed in the \textbf{normal mode}. To learn
more about Vim, please check this
\href{https://missing.csail.mit.edu/2020/editors/}{lecture}.

\hypertarget{sec:ssh}{%
\subsection{SSH}\label{sec:ssh}}

The Secure Shell (SSH) protocol is a method for securely sending
commands to a computer over an unsecured network. SSH uses cryptography
to authenticate and encrypt connections between devices. It is widely
used to: - push code to a remote git repository, - log into a remote
machine and execute commands.

With a host name (the IP of the target machine to login) and a user
name, one can use the following command to login,

\begin{lstlisting}[language=bash]
ssh <username>@<hostname>
\end{lstlisting}

where \passthrough{\lstinline!<username>!} is the user's account name
and \passthrough{\lstinline!<hostname>!} is the host name or IP of the
target machine. You will get logged in after inputting the password.

\textbf{Tips to make your life easier}

It will be tedious to type the host name and user name everytime you
want to login to the remote machine. You can setup the
\passthrough{\lstinline!\~/.ssh/config!} file to make your life easier.
The following is an example of the
\passthrough{\lstinline!\~/.ssh/config!} file.

\begin{lstlisting}
Host amat5315
  HostName <hostname>
  User <username>
\end{lstlisting}

where \passthrough{\lstinline!amat5315!} is the alias of the host. After
setting up the \passthrough{\lstinline!\~/.ssh/config!}, you can login
to the remote machine by typing

\begin{lstlisting}[language=bash]
ssh amat5315
\end{lstlisting}

If you want to avoid typing the password everytime you login, you can
use the command

\begin{lstlisting}[language=bash]
ssh-keygen
\end{lstlisting}

to generate a pair of public and private keys, which will be stored in
the \passthrough{\lstinline!\~/.ssh!} folder on the local machine. After
setting up the keys, you can copy the public key to the remote machine
by typing

\begin{lstlisting}[language=bash]
ssh-copy-id amat5315
\end{lstlisting}

Try connecting to the remote machine again, you will find that you don't
need to type the password anymore.

\textbf{How does SSH key pair work?} The SSH key pair is a pair of
asymmetric keys, one is the public key and the other is the private key.
In the above example, the public key is uploaded to the remote machine
and the private key is stored on the local machine. The public key can
be shared with anyone, but the private key must be kept secret.

To connect to a server, the server needs to know that you are the one
who with the right to access it. To do so, the server will need to check
if you have the private key that corresponds to the public key stored on
the server. If you have the private key, you will be granted access to
the server.

The secret of the SSH key pair is that \textbf{the public key can be
used to encrypt a message that can only be decrypted by the private
key}, i.e.~the public key is more like a lock and the private key is the
key to unlock the lock. This is the foundation of the SSH protocol. So
server can send you a message encrypted by your public key, and only you
can decrypt it with your private key. This is how the server knows that
you are the one who has the private key without actually sending the
private key to the server.

\hypertarget{sec:version-control}{%
\section{Maintainability - Version Control}\label{sec:version-control}}

Maintaining a software project is not easy, especially when it comes to
multiple developers working on the same piece of code. When adding a new
feature to the project, maintainers may encounter the following
problems:

\begin{itemize}
\tightlist
\item
  Multiple developers modify the same file at the same time, works can
  not be merged easily.
\item
  New code breaks an existing feature, downstream users are affected.
\end{itemize}

The solution to the above problems is \textbf{version-control}. Among
all version control software, \textbf{git} is the most popular one.

\hypertarget{create-a-git-repository}{%
\subsection{Create a git repository}\label{create-a-git-repository}}

A git repository, also known as a repo, is basically a directory where
your project lives and git keeps track of your file's history. To get
started, you start with a terminal and type

\begin{lstlisting}[language=bash]
cd path/to/working/directory
git init
echo "Hello, World" > README.md
git add -A
git commit -m 'this is my initial commit'
git status
\end{lstlisting}

\begin{itemize}
\tightlist
\item
  Line 1: changes the directory to the working directory, which can be
  either an existing directory or a new directory.
\item
  Line 2: initializes a git repository in the working directory. A
  \passthrough{\lstinline!.git!} directory is created in the working
  directory, which contains all the necessary metadata for the repo.
\item
  Line 3: creates a file \passthrough{\lstinline!README.md!} with the
  content \passthrough{\lstinline!Hello, World!}. The file
  \passthrough{\lstinline!README.md!} is a \textbf{markdown} file, which
  is a lightweight markup language with plain-text-formatting syntax.
  You can learn more about markdown from the
  \href{https://www.markdowntutorial.com/}{markdown tutorial}. This line
  can be omitted if the working directory already contains files.
\item
  Line 4: line add files to the \textbf{staging area} (area that caches
  changes that to be committed).
\item
  Line 5: commits the changes to the repository, which will create a
  \textbf{snapshot} of your current work.
\item
  Line 6: shows the status of the working directory, staging area, and
  repository. If the above commands are executed correctly, the output
  should be
  \passthrough{\lstinline!nothing to commit, working tree clean!}.
\end{itemize}

\hypertarget{track-the-changes}{%
\subsection{Track the changes}\label{track-the-changes}}

\textbf{Git} enables developers to track changes in their codebase.
Continuing the previous example, we can analyze the repository with the
following commands:

\begin{lstlisting}[language=bash]
echo "Bye Bye, World" > README.md
git diff
git add -A
git commit -m 'a second commit'
git log
git checkout HEAD~1
git checkout main
\end{lstlisting}

\begin{itemize}
\tightlist
\item
  Line 1: makes changes to the file \passthrough{\lstinline!README.md!}.
\item
  Line 2: shows the changes made to the file
  \passthrough{\lstinline!README.md!}.
\item
  Line 3-4: adds the changes to the staging area and commits the changes
  to the repository.
\item
  Line 5: shows the history of commits. The output should be something
  like this:
\end{itemize}

\begin{lstlisting}
commit 02cd535b6d78fca1713784c61eec86e67ce9010c (HEAD -> main)
Author: GiggleLiu <cacate0129@gmail.com>
Date:   Mon Feb 5 14:34:20 2024 +0800

    a second commit

commit 570e390759617a7021b0e069a3fbe612841b3e50
Author: GiggleLiu <cacate0129@gmail.com>
Date:   Mon Feb 5 14:23:41 2024 +0800

    this is my initial commit
\end{lstlisting}

\begin{itemize}
\tightlist
\item
  Line 6: Checkout the previous snapshot. Note
  \passthrough{\lstinline!HEAD!} is your current snapshot and
  \passthrough{\lstinline!HEAD\~n!} is the \passthrough{\lstinline!n!}th
  snapshot counting back from the current snapshot.
\item
  Line 7: Return to the \passthrough{\lstinline!main!} \textbf{branch},
  which points to the latest snapshot. We will discuss more about
  \textbf{branch} later in this tutorial.
\end{itemize}

You can use \passthrough{\lstinline!git reset!} to reset the current
HEAD to the specified snapshot, which can be useful when you committed
something bad by accident.

\hypertarget{work-with-remote-repositories}{%
\subsection{Work with remote
repositories}\label{work-with-remote-repositories}}

A server to store git repository, or \textbf{remote} in git terminology,
is required for the collaboration purpose. Remote repositories can be
hosted on git hosting services like GitHub, GitLab, or Bitbucket. After
creating a new empty repository (no README files) on a git hosting
service
(\href{https://docs.github.com/en/get-started/quickstart/create-a-repo}{How
to create a new github repo?}), a URL for cloning the repo will show up,
which that usually starts with \passthrough{\lstinline!git!} or
\passthrough{\lstinline!https!}. Let us denote this URL as
\passthrough{\lstinline!<url>!} and continue the previous example:

\begin{lstlisting}[language=bash]
git remote add origin <url>
git remote -v
git push origin main
\end{lstlisting}

\begin{itemize}
\tightlist
\item
  Line 1: add a remote repository, where
  \passthrough{\lstinline!origin!} is a tag for the added remote.
\item
  Line 2: shows the URL of all remotes, including the
  \passthrough{\lstinline!origin!} remote we just added.
\item
  Line 3: push commits to the \passthrough{\lstinline!main!} branch of
  the remote repository \passthrough{\lstinline!origin!}. This command
  sometimes could fail due to another commit pushed to the remote
  earlier, where the commit may from another machine or another person.
  To resolve the issue, you can use
  \passthrough{\lstinline!git pull origin main!} to fetch the latest
  snapshot on the remote. \passthrough{\lstinline!git pull!} may also
  fail, because the remote commit may be incompatible with the local
  commit, e.g.~the same file has been changed. In this worst case, you
  need to merge two commits manually (link).
\end{itemize}

\hypertarget{develop-features-safely---branches}{%
\subsection{Develop features safely -
branches}\label{develop-features-safely---branches}}

So far, we worked with a single branch \passthrough{\lstinline!main!}. A
\textbf{branch} in git is a lightweight pointer to a specific commit.
Working on a single branch is dangerous due to the following reasons: -
\emph{No usable code.} Developers usually develop features based on the
current \passthrough{\lstinline!main!} branch, so the
\passthrough{\lstinline!main!} branch is expected to always usable.
However, working on a single branch can easily break this rule. -
\emph{Hard to resolve conflicts.} when multiple developers modify the
same file at the same time, works can not be merged easily. Multiple
branches can make the feature development process independent of each
other, which can avoid conflicts. - \emph{Hard to discard a feature.}
For some experimental features, you may want to discard it after
testing. A commit on the main branch can not be easily reverted.

Understanding the branches is extremely useful when, multiple developers
are working on different features.

\begin{lstlisting}[language=bash]
git checkout -b me/feature
echo "Hello, World - Version 2" > README.md
git add -A
git commit -m 'this is my feature'
git push origin me/feature
\end{lstlisting}

\begin{itemize}
\tightlist
\item
  Line 1: create and switch to the new branch
  \passthrough{\lstinline!me/feature!}. Here, we use the branch name
  \passthrough{\lstinline!me/feature!} to indicate that this branch is
  for the feature developed by \passthrough{\lstinline!me!}, which is a
  matter of convention.
\item
  Line 2-5: makes some changes to the file
  \passthrough{\lstinline!README.md!} and commits the changes to the
  repository. Finally, the changes are pushed to the remote repository
  \passthrough{\lstinline!origin!}. The remote branch
  \passthrough{\lstinline!me/feature!} is created automatically.
\end{itemize}

While developing a feature, you or another developer may want to develop
another feature based on the current \passthrough{\lstinline!main!}
branch. You can create another branch
\passthrough{\lstinline!other/feature!} and develop the feature there.

\begin{lstlisting}[language=bash]
git checkout main
git checkout -b other/feature
echo "Bye Bye, World - Version 2" > feature.md
git add -A
git commit -m 'this is another feature'
git push origin other/feature
\end{lstlisting}

In the above example, we created a new branch
\passthrough{\lstinline!other/feature!} based on the
\passthrough{\lstinline!main!} branch, and made some changes to the file
\passthrough{\lstinline!feature.md!}.

Finally, when the feature is ready, you can merge the feature branch to
the main branch.

\begin{lstlisting}[language=bash]
git checkout main
git merge me/feature
git push origin main
\end{lstlisting}

\hypertarget{working-with-others---issues-and-pull-requests}{%
\subsection{Working with others - issues and pull
requests}\label{working-with-others---issues-and-pull-requests}}

When working with others, you may want to propose changes to a
repository and discuss them with others. This is where \textbf{issues}
and \textbf{pull requests} come in. Issues and pull requests are
features of git hosting services like GitHub and GitLab. -
\textbf{Issue} is relatively simple, it is a way to report a bug or
request a feature. - \textbf{Pull request} (resource:
\href{https://docs.github.com/en/pull-requests/collaborating-with-pull-requests/proposing-changes-to-your-work-with-pull-requests/creating-a-pull-request}{how
to create a pull request}) is a way to propose changes to a repository
and discuss them with others. It is also a way to merge code from source
branch to target branch. The source branch can be a branch in the same
repository or a branch in a \textbf{forked repository} - a copy of the
repository in your account. Forking a repository is needed when you want
to propose changes to a repository that you do not have write access to.

\begin{quote}
\textbf{FAQ: Should I make a pull requests or push directly to main
branch?}

To update the main branch, one should use pull requests as much as
possible, even if you have write access to the repository. It is a good
practice to discuss the changes with others before merging them to the
main branch. A pull request also makes the changes more traceable, which
is useful when you want to revert the changes.
\end{quote}

\hypertarget{summary-a-cheat-sheet}{%
\subsection{Summary: a cheat sheet}\label{summary-a-cheat-sheet}}

It is not possible to cover all the feature of git. We will list a few
useful commands and resources for git learning.

\begin{lstlisting}
# global config
git config  # Get and set repository or global options

# initialize a repo
git init    # Create an empty Git repo or reinitialize an existing one
git clone   # Clone repository into new directory

# info
git status  # Show the working tree status
git log     # Show commit logs
git diff    # Show changes between commits, commit and working tree, etc

# work on a branch
git add     # Add file contents to the index
git rm      # Remove files from the working tree and from the index
git commit  # Record changes to the repository
git reset   # Reset current HEAD to the specified state

# branch manipulation
git checkout # Switch branches or restore working tree files
git branch  # List, create, or delete branches
git merge   # Join two or more development histories together

# remote synchronization
git remote  # Manage set of tracked repositories
git pull  # Fetch from and integrate with another repo or a local branch
git fetch   # Download objects and refs from another repository
git push    # Update remote refs along with associated objects
\end{lstlisting}

\hypertarget{resources}{%
\subsection{Resources}\label{resources}}

\begin{itemize}
\tightlist
\item
  \href{https://githubtraining.github.io/training-manual/book.pdf}{The
  Official GitHub Training Manual}
\item
  MIT online course \href{https://missing.csail.mit.edu/2020/}{missing
  semester}.
\end{itemize}

\hypertarget{sec:ci-cd}{%
\section{Correctness - Unit Tests}\label{sec:ci-cd}}

In terms of scientific computing, accuracy of your result is most
certainly more important than anything else. To ensure the correctness
of the code, we employ two methods: \textbf{Unit Testing} and
\textbf{CI/CD}.

\hypertarget{unit-test}{%
\subsection{Unit Test}\label{unit-test}}

Unit tests are typically
\href{https://en.wikipedia.org/wiki/Automated_test}{automated tests}
written and run by
\href{https://en.wikipedia.org/wiki/Software_developer}{software
developers} to ensure that a section of an application (known as the
``unit'') meets its
\href{https://en.wikipedia.org/wiki/Software_design}{design} and behaves
as intended. Unit tests are composed of a series of individual test
cases, each of which verifies the correctness by using
\textbf{assertions}. If all assertions are true, the test case passes;
otherwise, it fails. The unit tests are run automatically whenever the
code is changed, ensuring that the code is always in a working state. In
Julia, there exists a helpful module called
\href{https://docs.julialang.org/en/v1/stdlib/Test/}{Test} to help you
do unit testing.

\hypertarget{automate-your-workflow---cicd}{%
\subsection{Automate your workflow -
CI/CD}\label{automate-your-workflow---cicd}}

CI/CD, which stands for continuous integration and continuous
delivery/deployment, aims to streamline and accelerate the software
development lifecycle. CI/CD are often integrated with git hosting
services, e.g.~\href{https://docs.github.com/en/actions}{Github
Actions}. Typical CI/CD pipelines include the following steps:

\begin{itemize}
\tightlist
\item
  Automatically \textbf{build}, \textbf{test} and \textbf{merge} the
  code changes whenever a developer commits code to the repository.
\item
  Automatically \textbf{deploy} the code or documentation to a cloud
  service.
\end{itemize}

The CI/CD pipeline is a powerful tool to ensure the correctness of the
code and the reproducibility of the results. It is also a good practice
to use CI/CD to automate the workflow, especially when you are working
with a team.

\hypertarget{sec:julia}{%
\chapter{Julia}\label{sec:julia}}

\passthrough{\lstinline!Julia!} is a high-level, high-performance,
dynamic programming language. From the designing stage,
\passthrough{\lstinline!Julia!} is intended to address the needs of
high-performance numerical analysis and computational science, without
the typical need of separate compilation to be fast, while also being
effective for general-purpose programming, web use or as a specification
language. \passthrough{\lstinline!Julia!} is also a free and open-source
language, with a \href{https://julialang.org/community/}{large
community} and a \href{https://juliahub.com/}{rich ecosystem}.

We will devlve deeper into \passthrough{\lstinline!Julia!} later in the
chapter. For now, we will just install \passthrough{\lstinline!Julia!}
and setup the environment.

\hypertarget{sec:setup}{%
\section{Setup Julia}\label{sec:setup}}

\hypertarget{step-1-installing-julia}{%
\subsection{Step 1: Installing Julia}\label{step-1-installing-julia}}

For Linux/Mac users, please open a terminal and type the following
command to install \href{https://julialang.org/}{Julia} with
\href{https://github.com/JuliaLang/juliaup}{juliaup}.
\passthrough{\lstinline!Juliaup!} is a tool to manage Julia versions and
installations. It allows you to install multiple versions of Julia and
switch between them easily.

\begin{lstlisting}[language=bash]
curl -fsSL https://install.julialang.org | sh # Linux and macOS
\end{lstlisting}

For Windows users, please open execute the following command in a
\passthrough{\lstinline!cmd!},

\begin{lstlisting}
winget install julia -s msstore # Windows
\end{lstlisting}

You can also install Juliaup directly from
\href{https://www.microsoft.com/store/apps/9NJNWW8PVKMN}{Windows Store}.

\hypertarget{for-users-suffering-from-the-slow-download-speed}{%
\subsection{For users suffering from the slow download
speed}\label{for-users-suffering-from-the-slow-download-speed}}

Network connectivity can be an issue for some users, especially for
those who are in China. You may need to specify another server for
installing Juliaup and Julia packages. To do so, execute the following
command in your terminal before running the script above.

\textbf{Linux and macOS}

\begin{lstlisting}[language=bash]
export JULIAUP_SERVER=https://mirror.nju.edu.cn/julia-releases/ # Linux & macOS
export JULIA_PKG_SERVER=https://mirrors.nju.edu.cn/julia
\end{lstlisting}

\textbf{Windows}

\begin{lstlisting}
$env:JULIAUP_SERVER="https://mirror.nju.edu.cn/julia-releases/" # Windows
$env:JULIA_PKG_SERVER="https://mirrors.nju.edu.cn/julia"
\end{lstlisting}

An alternative approach is downloading the corresponding Julia binary
from the \href{https://mirror.nju.edu.cn/julia-releases/}{Nanjing
university mirror website}. After installing the binary, please set the
Julia binary path properly if you want to start a Julia REPL from a
terminal, check this
\href{https://julialang.org/downloads/platform/}{manual page} to learn
more.

\hypertarget{installing-julia}{%
\subsection{Installing Julia}\label{installing-julia}}

To verify that Julia is installed, please open a \textbf{new} terminal
and run the following command in your terminal.
\passthrough{\lstinline!bash   julia!} - It should start a Julia
REPL(Read-Eval-Print-Loop) session like this - If you wish to install a
specific version of Julia, please refer to the
\href{https://github.com/JuliaLang/juliaup}{documentation}.

\hypertarget{step-2-package-management}{%
\subsection{Step 2: Package
Management}\label{step-2-package-management}}

\begin{itemize}
\tightlist
\item
  \passthrough{\lstinline!Julia!} has a mature eco-system for scientific
  computing.
\item
  \passthrough{\lstinline!Pkg!} is the built-in package manager for
  Julia.
\item
  To enter the package manager, press \passthrough{\lstinline!]!} in the
  REPL.
  \includegraphics{https://github.com/exAClior/QMBCTutorial/blob/ys/julia-tutorial/notebooks/resources/scripts/Packages.gif?raw=true}
\item
  The environment is indicated by the \passthrough{\lstinline!(@v1.9)!}.
\item
  To add a package, type \passthrough{\lstinline!add <package name>!}.
\item
  To exit the package manager press \passthrough{\lstinline!backspace!}
  key
\item
  \href{https://pkgdocs.julialang.org/v1/managing-packages/}{Read More}
\end{itemize}

\hypertarget{step-3.-configure-the-startup-file}{%
\subsection{Step 3. Configure the startup
file}\label{step-3.-configure-the-startup-file}}

First create a new file
\passthrough{\lstinline!\~/.julia/config/startup.jl!} by executing the
following commands

\passthrough{\lstinline!mkdir -r \~/.julia/config!}
\passthrough{\lstinline!touch \~/.julia/config/startup.jl!}

You could open the file with your favourite editor and add the following
content

\begin{lstlisting}
try
    using Revise
catch e
    @warn "fail to load Revise."
end
\end{lstlisting}

The contents in the startup file is executed immediately after you open
a new Julia session.

Then you need to install
\href{https://github.com/timholy/Revise.jl}{Revise}, which is an Julia
package that can greatly improve the using experience of Julia. To
install \passthrough{\lstinline!Revise!}, open Julia REPL and type

\begin{lstlisting}
julia> using Pkg; Pkg.add("Revise")
\end{lstlisting}

If you don't know about \passthrough{\lstinline!startup.jl!} and where
to find it,
\href{https://docs.julialang.org/en/v1/manual/command-line-interface/\#Startup-file}{here}
is a good place for further information.

\hypertarget{more-packages}{%
\subsection{More Packages}\label{more-packages}}

\begin{itemize}
\tightlist
\item
  You may find more Julia packages \href{https://juliahub.com/}{here}.
\end{itemize}

As a final step, please verify your Julia configuration by openning a
Julia REPL and type

\begin{lstlisting}
julia> versioninfo()
Julia Version 1.9.2
Commit e4ee485e909 (2023-07-05 09:39 UTC)
Platform Info:
  OS: macOS (arm64-apple-darwin22.4.0)
  CPU: 10 × Apple M2 Pro
  WORD_SIZE: 64
  LIBM: libopenlibm
  LLVM: libLLVM-14.0.6 (ORCJIT, apple-m1)
  Threads: 1 on 6 virtual cores
Environment:
  JULIA_NUM_THREADS = 1
  JULIA_PROJECT = @.
  JULIA_PKG_SERVER = http://cn-southeast.pkg.juliacn.com/ 
\end{lstlisting}

\hypertarget{step-4.-download-an-editor-vscode}{%
\subsection{Step 4. Download an editor:
VSCode}\label{step-4.-download-an-editor-vscode}}

Install VSCode by downloading the correct binary for your platform from
\href{https://code.visualstudio.com/download}{here}. Open VSCode and
open the \passthrough{\lstinline!Extensions!} tab on the left side-bar
of the window, search \passthrough{\lstinline!Julia!} and install the
most popular extension.
\href{https://github.com/julia-vscode/julia-vscode}{read more\ldots{}}

You are ready to go, cheers!

\hypertarget{the-four-modes-of-julia-repl}{%
\subsection{The four modes of Julia
REPL}\label{the-four-modes-of-julia-repl}}

A Julia REPL has four modes,

\begin{enumerate}
\def\labelenumi{\arabic{enumi}.}
\item
  \textbf{Julian mode} is the default mode that can interpret your Julia
  code.
\item
  \textbf{Shell mode} is the mode that you can run shell commands. Press
  \passthrough{\lstinline!;!} in the Julian mode and type
\end{enumerate}

\begin{lstlisting}
shell> date
Sun Nov  6 10:50:21 PM CST 2022
\end{lstlisting}

To return to the Julian mode, type the Backspace key.

\begin{enumerate}
\def\labelenumi{\arabic{enumi}.}
\setcounter{enumi}{2}
\tightlist
\item
  \textbf{Package mode} is the mode that you can manage packages. Press
  \passthrough{\lstinline!]!} in the Julian mode and type
\end{enumerate}

\begin{lstlisting}
(@v1.8) pkg> st
Status `~/.julia/environments/v1.8/Project.toml`
  [295af30f] Revise v3.4.0
\end{lstlisting}

To return to the Julian mode, type the Backspace key.

\begin{enumerate}
\def\labelenumi{\arabic{enumi}.}
\setcounter{enumi}{3}
\tightlist
\item
  \textbf{Help mode} is the mode that you can access the docstrings of
  functions. Press \passthrough{\lstinline!?!} in the Julian mode and
  type
\end{enumerate}

\begin{lstlisting}
help> sum
... docstring for sum ...
\end{lstlisting}

To return to the Julian mode, type the Backspace key.

\href{https://docs.julialang.org/en/v1/stdlib/REPL/}{read more\ldots{}}

\hypertarget{sec:whyjulia}{%
\section{Why Julia is fast?}\label{sec:whyjulia}}

\hypertarget{what-is-julia-programming-language}{%
\subsection{What is Julia programming
language?}\label{what-is-julia-programming-language}}

Julia is a modern, open-source, high performance programming language
for technical computing. It was born in 2012 in MIT, now is maintained
by JuliaHub Inc.~located in Boston, US.

\emph{Julia is open-source.} Julia source code is maintained on GitHub
repo \href{https://github.com/JuliaLang/julia}{JuliaLang/julia}, and it
open-source LICENSE is MIT. Julia packages can be found on
\href{https://juliahub.com/ui/Packages}{JuliaHub}, most of them are
open-source.

\emph{Julia is designed for high performance}
(\href{https://arxiv.org/abs/1209.5145}{arXiv:1209.5145}). It is a
dynamic programming language, but it is as fast as C/C++. The following
figure shows the computing time of multiple programming languages
normalized to C/C++.

\emph{Julia is a trend in scientific computing.} Many famous scientists
and engineers have switched to Julia from other programming languages.

\begin{itemize}
\tightlist
\item
  \textbf{Steven G. Johnson}, creater of
  \href{http://www.fftw.org/}{FFTW}, switched from C++ to Julia years
  ago.
\item
  \textbf{Anders Sandvik}, creater of Stochastic Series Expansion (SSE)
  quantum Monte Carlo method, switched from Fortran to Julia recently.

  \begin{itemize}
  \tightlist
  \item
    Course link: \href{https://physics.bu.edu/~py502/}{Computational
    Physics}
  \end{itemize}
\item
  \textbf{Miles Stoudenmire}, creater of
  \href{https://itensor.org/}{ITensor}, switched from C++ to Julia years
  ago.
\item
  \textbf{Jutho Haegeman}, \textbf{Chris Rackauckas} and more.
\end{itemize}

\begin{quote}
\textbf{FAQ: Should I switch to Julia?}

Before switching to Julia, please make sure:

\begin{itemize}
\tightlist
\item
  the problem you are trying to solve runs more than 10min.
\item
  you are not satisfied by any existing tools.
\end{itemize}
\end{quote}

\hypertarget{my-first-program-factorial}{%
\subsection{My first program:
Factorial}\label{my-first-program-factorial}}

Before we start, please make sure you have the needed packages
installed. Type \passthrough{\lstinline!]!} in the Julia REPL to enter
the package manager, and then type

\begin{lstlisting}
pkg> add BenchmarkTools, MethodAnalysis
\end{lstlisting}

Go back to the REPL by pressing \passthrough{\lstinline!Backspace!}.

\begin{lstlisting}
julia> function jlfactorial(n)
           x = 1
           for i in 1:n
               x = x * i
           end
           return x
       end
jlfactorial (generic function with 1 method)
\end{lstlisting}

To make sure the performance is measured correctly, we use the
\passthrough{\lstinline!@btime!} macro in the
\passthrough{\lstinline!BenchmarkTools!} package to measure the
performance of the function.

\begin{lstlisting}
julia> @btime jlfactorial(x) setup=(x=5)
2.208 ns (0 allocations: 0 bytes)
120
\end{lstlisting}

CPU clock cycle is \textasciitilde0.3ns, so it takes only a few clock
cycles to compute the factorial of 5. Julia is really fast!

\hypertarget{compare-with-the-speed-of-c-program}{%
\subsection{Compare with the speed of C
program}\label{compare-with-the-speed-of-c-program}}

To measure the performance of the C program, we can utilize the
benchmark utilities in Julia. Benchmarking C program with Julia is
accurate because Julia has perfect interoperability with C, which allows
zero-cost calling of C functions.

In the following example, we first write a C program to calculate the
factorial of a number. The file is named
\passthrough{\lstinline!demo.c!}, and the content is as follows:

\begin{lstlisting}[language=bash]
$ cat demo.c
#include <stddef.h>
int c_factorial(size_t n) {
    int s = 1;
    for (size_t i=1; i<=n; i++) {
        s *= i;
    }
    return s;
}
\end{lstlisting}

To execute a C program in Julia, one needs to compile it to a shared
library.

\begin{lstlisting}
$ gcc demo.c -fPIC -O3 -shared -o demo.so
\end{lstlisting}

To call the function in Julia, one can use the
\passthrough{\lstinline!@ccall!} macro in the
\passthrough{\lstinline!Libdl!} package
(\href{https://docs.julialang.org/en/v1/manual/calling-c-and-fortran-code/}{learn
more}). Please open a Julia REPL and execute the following code:

\begin{lstlisting}
julia> using Libdl

julia> c_factorial(x) = Libdl.@ccall "./demo.so".c_factorial(x::Csize_t)::Int
\end{lstlisting}

The benchmark result is as follows:

\begin{lstlisting}
julia> using BenchmarkTools

julia> @benchmark c_factorial(5)
BenchmarkTools.Trial: 10000 samples with 1000 evaluations.
 Range (min … max):  7.333 ns … 47.375 ns  ┊ GC (min … max): 0.00% … 0.00%
 Time  (median):     7.458 ns              ┊ GC (median):    0.00%
 Time  (mean ± σ):   7.764 ns ±  1.620 ns  ┊ GC (mean ± σ):  0.00% ± 0.00%

  ██▅  ▃▁ ▂▂                         ▁▁▁                     ▂
  ███▆▄██▆███▅▅▆▆▆▅▆▅▄▅▆▅▅▇▆▆▄▅▅▇█▇▆▆█████▅▃▁▁▁▁▁▁▁▃▁▁▁▁▁▁▁▃ █
  7.33 ns      Histogram: log(frequency) by time     12.6 ns <

 Memory estimate: 0 bytes, allocs estimate: 0.
\end{lstlisting}

Although the C program requires the type of variables to be manually
declared, its performance is very good. The computing time is only 7.33
ns.

\hypertarget{compare-with-the-speed-of-python-program}{%
\subsection{Compare with the speed of Python
program}\label{compare-with-the-speed-of-python-program}}

We use the \passthrough{\lstinline!timeit!} module in ipython to measure
the performance of the Python program.

\begin{lstlisting}
In [5]: def factorial(n):
...:        x = 1
...:        for i in range(1, n+1):
...:            x = x * i
...:        return x
...:

In [6]: factorial(5)
Out[6]: 120

In [7]: timeit factorial(5)
144 ns ± 0.379 ns per loop (mean ± std. dev. of 7 runs, 10,000,000 loops each)

In [8]: factorial(100)
Out[8]: 93326215443944152681699238856266700490715968264381621468592963895217599993229915608941463976156518286253697920827223758251185210916864000000000000000000000000
\end{lstlisting}

One can also use the \passthrough{\lstinline!PyCall!} package to call
the Python function in Julia.

The computing time of the Python program is 144 ns, which is 20 times
slower than the C program and 70 times slower than the Julia program. On
the other hand, the python program is more flexible since its integer
type is not limited by the machine word size.

\begin{lstlisting}
julia> typemax(Int)
9223372036854775807

julia> jlfactorial(100)
0
\end{lstlisting}

The reason why python is slow and flexible are the same. In python the
type of a variable is not declared when it is defined, and it can be
changed at any time. This is why the integer type becomes an arbitrary
precision integer type when the number is too large. If a variable does
not have a fixed type, the program can not preallocate memory for it due
to the lack of size information. Then a dynamic typed language has to
use a tuple \passthrough{\lstinline!(type, *data)!} to represent an
object, where \passthrough{\lstinline!type!} is the type of the object
and \passthrough{\lstinline!*data!} is the pointer to the data. Pointing
to a random memory location is slow, because it violates the principle
of data locality. Lacking of data locality causes the frequent cache
miss - failure to find the data in the L1, L2, or L3 cache. Loading data
from the main memory is slow, because of the long latency of reading the
main memory.

\hypertarget{combining-python-and-cc}{%
\subsection{Combining Python and C/C++?}\label{combining-python-and-cc}}

From the maintainer's perspective, it is hard to maintain a program
written in both Python and C/C++: - It makes the build configuration
files complicated. - Learning two programming languages is hard for new
contributors.

Using python as glue is not as powerful as it looks, the following
problem can not be solved by this approach: - Monte Carlo simulation. -
Branching and bound algorithms.

\hypertarget{julias-solution-just-in-time-jit-compilation}{%
\subsection{Julia's solution: Just in time (JIT)
compilation}\label{julias-solution-just-in-time-jit-compilation}}

Given that you have defined a Julia function, the Julia compiler will
generate a binary for the function when it is called for the first time.
The binary is called a \textbf{method instance}, and it is generated
based on the \textbf{input types} of the function. The method instance
is then stored in the method table, and it will be called when the
function is called with the same input types. The method instance is
generated by the LLVM compiler, and it is optimized for the input types.
The method instance is a binary, and it is as fast as a C/C++ program.

\textbf{Step 1: Infer the types}

\begin{lstlisting}
julia> @code_warntype jlfactorial(10)
MethodInstance for jlfactorial(::Int64)
  from jlfactorial(n) @ Main REPL[4]:1
Arguments
  #self#::Core.Const(jlfactorial)
  n::Int64
Locals
  @_3::Union{Nothing, Tuple{Int64, Int64}}
  x::Int64
  i::Int64
Body::Int64
1 ─       (x = 1)
│   %2  = (1:n)::Core.PartialStruct(UnitRange{Int64}, Any[Core.Const(1), Int64])
│         (@_3 = Base.iterate(%2))
│   %4  = (@_3 === nothing)::Bool
│   %5  = Base.not_int(%4)::Bool
└──       goto #4 if not %5
2 ┄ %7  = @_3::Tuple{Int64, Int64}
│         (i = Core.getfield(%7, 1))
│   %9  = Core.getfield(%7, 2)::Int64
│         (x = x * i)
│         (@_3 = Base.iterate(%2, %9))
│   %12 = (@_3 === nothing)::Bool
│   %13 = Base.not_int(%12)::Bool
└──       goto #4 if not %13
3 ─       goto #2
4 ┄       return x
\end{lstlisting}

When type inference fails

\begin{lstlisting}
julia> badcode(x) = x > 3 ? 1.0 : 3

julia> @code_warntype badcode(4)
MethodInstance for badcode(::Int64)
  from badcode(x) @ Main REPL[9]:1
Arguments
  #self#::Core.Const(badcode)
  x::Int64
Body::Union{Float64, Int64}
1 ─ %1 = (x > 3)::Bool
└──      goto #3 if not %1
2 ─      return 1.0
3 ─      return 3
\end{lstlisting}

\passthrough{\lstinline!Union\{Float64, Int64\}!} means the return type
is either \passthrough{\lstinline!Float64!} or
\passthrough{\lstinline!Int64!}.

Type unstable code is slow

\begin{lstlisting}
julia> x = rand(1:10, 1000);

julia> @benchmark badcode.($x)
BenchmarkTools.Trial: 10000 samples with 8 evaluations.
 Range (min … max):  2.927 μs … 195.198 μs  ┊ GC (min … max):  0.00% … 96.52%
 Time  (median):     3.698 μs               ┊ GC (median):     0.00%
 Time  (mean ± σ):   4.257 μs ±   7.894 μs  ┊ GC (mean ± σ):  12.43% ±  6.54%

                 ▁▅█▅▃▂                                        
  ▁▃▅▇▇▇▅▃▂▂▂▃▄▆▇███████▇▇▅▄▄▃▃▃▃▃▃▂▂▃▂▂▂▂▁▂▂▁▁▁▁▁▁▁▁▁▁▁▁▁▁▁▁ ▃
  2.93 μs         Histogram: frequency by time        5.44 μs <

 Memory estimate: 26.72 KiB, allocs estimate: 696.

julia> stable(x) = x > 3 ? 1.0 : 3.0
stable (generic function with 1 method)

julia> @benchmark stable.($x)
BenchmarkTools.Trial: 10000 samples with 334 evaluations.
 Range (min … max):  213.820 ns … 25.350 μs  ┊ GC (min … max):  0.00% … 98.02%
 Time  (median):     662.551 ns              ┊ GC (median):     0.00%
 Time  (mean ± σ):   947.978 ns ±  1.187 μs  ┊ GC (mean ± σ):  29.30% ± 21.05%

  ▂▃▅██▇▅▄▃▂▁                                                  ▂
  ████████████▇▅▅▄▄▁▁▁▁▁▁▁▁▁▁▁▁▁▃▅▆▇██████▇▇▇▆█▇▇▇▇▇▇▇▇▆▇▆▆▆▇▇ █
  214 ns        Histogram: log(frequency) by time      6.32 μs <

 Memory estimate: 7.94 KiB, allocs estimate: 1.
\end{lstlisting}

\begin{itemize}
\tightlist
\item
  ``\passthrough{\lstinline!.!}'' is the broadcasting operator.
\item
  ``\passthrough{\lstinline!$!}'' is the interpolation operator, it is
  used to interpolate a variable into an expression.
\end{itemize}

\textbf{Step 2: Generates the LLVM intermediate representation}

LLVM is a set of compiler and toolchain technologies that can be used to
develop a front end for any programming language and a back end for any
instruction set architecture. LLVM is the backend of multiple languages,
including Julia, Rust, Swift and Kotlin.

\begin{lstlisting}
julia> @code_llvm jlfactorial(10)

or any instruction set architecture. LLVM is the backend of multiple languages, including Julia, Rust, Swift and Kotlin.



;  @ /Users/liujinguo/jcode/ModernScientificComputing2024/Lecture2/3.julia.jl#==#d2429055-58e9-4d84-894f-2e639723e078:1 within `jlfactorial`
define i64 @julia_jlfactorial_3677(i64 signext %0) #0 {
top:
;  @ /Users/liujinguo/jcode/ModernScientificComputing2024/Lecture2/3.julia.jl#==#d2429055-58e9-4d84-894f-2e639723e078:3 within `jlfactorial`
; ┌ @ range.jl:5 within `Colon`
; │┌ @ range.jl:403 within `UnitRange`
; ││┌ @ range.jl:414 within `unitrange_last`
     %1 = call i64 @llvm.smax.i64(i64 %0, i64 0)
; └└└
; ┌ @ range.jl:897 within `iterate`
; │┌ @ range.jl:672 within `isempty`
; ││┌ @ operators.jl:378 within `>`
; │││┌ @ int.jl:83 within `<`
      %2 = icmp slt i64 %0, 1
; └└└└
  br i1 %2, label %L32, label %L17.preheader

L17.preheader:                                    ; preds = %top
;  @ /Users/liujinguo/jcode/ModernScientificComputing2024/Lecture2/3.julia.jl#==#d2429055-58e9-4d84-894f-2e639723e078:5 within `jlfactorial`
  %min.iters.check = icmp ult i64 %1, 2
  br i1 %min.iters.check, label %scalar.ph, label %vector.ph

vector.ph:                                        ; preds = %L17.preheader
  %n.vec = and i64 %1, 9223372036854775806
  %ind.end = or i64 %1, 1
  br label %vector.body

vector.body:                                      ; preds = %vector.body, %vector.ph
  %index = phi i64 [ 0, %vector.ph ], [ %induction12, %vector.body ]
  %vec.phi = phi i64 [ 1, %vector.ph ], [ %3, %vector.body ]
  %vec.phi11 = phi i64 [ 1, %vector.ph ], [ %4, %vector.body ]
  %offset.idx = or i64 %index, 1
  %induction12 = add i64 %index, 2
;  @ /Users/liujinguo/jcode/ModernScientificComputing2024/Lecture2/3.julia.jl#==#d2429055-58e9-4d84-894f-2e639723e078:4 within `jlfactorial`
; ┌ @ int.jl:88 within `*`
   %3 = mul i64 %vec.phi, %offset.idx
   %4 = mul i64 %vec.phi11, %induction12
   %5 = icmp eq i64 %induction12, %n.vec
   br i1 %5, label %middle.block, label %vector.body

middle.block:                                     ; preds = %vector.body
; └
;  @ /Users/liujinguo/jcode/ModernScientificComputing2024/Lecture2/3.julia.jl#==#d2429055-58e9-4d84-894f-2e639723e078:5 within `jlfactorial`
  %bin.rdx = mul i64 %4, %3
  %cmp.n = icmp eq i64 %1, %n.vec
  br i1 %cmp.n, label %L32, label %scalar.ph

scalar.ph:                                        ; preds = %middle.block, %L17.preheader
  %bc.resume.val = phi i64 [ %ind.end, %middle.block ], [ 1, %L17.preheader ]
  %bc.merge.rdx = phi i64 [ %bin.rdx, %middle.block ], [ 1, %L17.preheader ]
  br label %L17

L17:                                              ; preds = %L17, %scalar.ph
  %value_phi4 = phi i64 [ %7, %L17 ], [ %bc.resume.val, %scalar.ph ]
  %value_phi6 = phi i64 [ %6, %L17 ], [ %bc.merge.rdx, %scalar.ph ]
;  @ /Users/liujinguo/jcode/ModernScientificComputing2024/Lecture2/3.julia.jl#==#d2429055-58e9-4d84-894f-2e639723e078:4 within `jlfactorial`
; ┌ @ int.jl:88 within `*`
   %6 = mul i64 %value_phi6, %value_phi4
; └
;  @ /Users/liujinguo/jcode/ModernScientificComputing2024/Lecture2/3.julia.jl#==#d2429055-58e9-4d84-894f-2e639723e078:5 within `jlfactorial`
; ┌ @ range.jl:901 within `iterate`
; │┌ @ promotion.jl:521 within `==`
    %.not = icmp eq i64 %value_phi4, %1
; │└
   %7 = add nuw i64 %value_phi4, 1
; └
  br i1 %.not, label %L32, label %L17

L32:                                              ; preds = %L17, %middle.block, %top
  %value_phi10 = phi i64 [ 1, %top ], [ %bin.rdx, %middle.block ], [ %6, %L17 ]
;  @ /Users/liujinguo/jcode/ModernScientificComputing2024/Lecture2/3.julia.jl#==#d2429055-58e9-4d84-894f-2e639723e078:6 within `jlfactorial`
  ret i64 %value_phi10
}
\end{lstlisting}

\textbf{Step 3: Compiles to binary code}

\begin{lstlisting}
julia> @code_native jlfactorial(10)
    .section    __TEXT,__text,regular,pure_instructions
    .build_version macos, 14, 0
    .globl  _julia_jlfactorial_3726         ; -- Begin function julia_jlfactorial_3726
    .p2align    2
_julia_jlfactorial_3726:                ; @julia_jlfactorial_3726
; ┌ @ /Users/liujinguo/jcode/ModernScientificComputing2024/Lecture2/3.julia.jl#==#d2429055-58e9-4d84-894f-2e639723e078:1 within `jlfactorial`
; %bb.0:                                ; %top
; │ @ /Users/liujinguo/jcode/ModernScientificComputing2024/Lecture2/3.julia.jl#==#d2429055-58e9-4d84-894f-2e639723e078:3 within `jlfactorial`
; │┌ @ range.jl:5 within `Colon`
; ││┌ @ range.jl:403 within `UnitRange`
; │││┌ @ range.jl:414 within `unitrange_last`
    cmp x0, #0
    csel    x9, x0, xzr, gt
; │└└└
    cmp x0, #1
    b.lt    LBB0_3
; %bb.1:                                ; %L17.preheader
; │ @ /Users/liujinguo/jcode/ModernScientificComputing2024/Lecture2/3.julia.jl#==#d2429055-58e9-4d84-894f-2e639723e078:5 within `jlfactorial`
    cmp x9, #2
    b.hs    LBB0_4
; %bb.2:
    mov w8, #1
    mov w0, #1
    b   LBB0_7
LBB0_3:
    mov w0, #1
; │ @ /Users/liujinguo/jcode/ModernScientificComputing2024/Lecture2/3.julia.jl#==#d2429055-58e9-4d84-894f-2e639723e078:6 within `jlfactorial`
    ret
LBB0_4:                                 ; %vector.ph
    mov x12, #0
; │ @ /Users/liujinguo/jcode/ModernScientificComputing2024/Lecture2/3.julia.jl#==#d2429055-58e9-4d84-894f-2e639723e078:5 within `jlfactorial`
    and x10, x9, #0x7ffffffffffffffe
    orr x8, x9, #0x1
    mov w11, #1
    mov w13, #1
LBB0_5:                                 ; %vector.body
                                        ; =>This Inner Loop Header: Depth=1
; │ @ /Users/liujinguo/jcode/ModernScientificComputing2024/Lecture2/3.julia.jl#==#d2429055-58e9-4d84-894f-2e639723e078:4 within `jlfactorial`
; │┌ @ int.jl:88 within `*`
    madd    x11, x11, x12, x11
; │└
; │ @ /Users/liujinguo/jcode/ModernScientificComputing2024/Lecture2/3.julia.jl#==#d2429055-58e9-4d84-894f-2e639723e078:5 within `jlfactorial`
    add x14, x12, #2
; │ @ /Users/liujinguo/jcode/ModernScientificComputing2024/Lecture2/3.julia.jl#==#d2429055-58e9-4d84-894f-2e639723e078:4 within `jlfactorial`
; │┌ @ int.jl:88 within `*`
    mul x13, x13, x14
    mov x12, x14
    cmp x10, x14
    b.ne    LBB0_5
; %bb.6:                                ; %middle.block
; │└
; │ @ /Users/liujinguo/jcode/ModernScientificComputing2024/Lecture2/3.julia.jl#==#d2429055-58e9-4d84-894f-2e639723e078:5 within `jlfactorial`
    mul x0, x13, x11
    cmp x9, x10
    b.eq    LBB0_9
LBB0_7:                                 ; %L17.preheader15
    add x9, x9, #1
LBB0_8:                                 ; %L17
                                        ; =>This Inner Loop Header: Depth=1
; │ @ /Users/liujinguo/jcode/ModernScientificComputing2024/Lecture2/3.julia.jl#==#d2429055-58e9-4d84-894f-2e639723e078:4 within `jlfactorial`
; │┌ @ int.jl:88 within `*`
    mul x0, x0, x8
; │└
; │ @ /Users/liujinguo/jcode/ModernScientificComputing2024/Lecture2/3.julia.jl#==#d2429055-58e9-4d84-894f-2e639723e078:5 within `jlfactorial`
; │┌ @ range.jl:901 within `iterate`
    add x8, x8, #1
; │└
    cmp x9, x8
    b.ne    LBB0_8
LBB0_9:                                 ; %L32
; │ @ /Users/liujinguo/jcode/ModernScientificComputing2024/Lecture2/3.julia.jl#==#d2429055-58e9-4d84-894f-2e639723e078:6 within `jlfactorial`
    ret
; └
                                        ; -- End function
.subsections_via_symbols
\end{lstlisting}

\begin{lstlisting}
julia> methods(jlfactorial)
# 1 method for generic function "jlfactorial" from Main:
 [1] jlfactorial(n)
     @ REPL[4]:1
\end{lstlisting}

\begin{lstlisting}
julia> using MethodAnalysis

julia> methodinstances(jlfactorial)
1-element Vector{Core.MethodInstance}:
 MethodInstance for jlfactorial(::Int64)
\end{lstlisting}

Single method, multiple instances

\begin{lstlisting}
julia> jlfactorial(UInt32(5))
120

julia> methodinstances(jlfactorial)
2-element Vector{Core.MethodInstance}:
 MethodInstance for jlfactorial(::Int64)
 MethodInstance for jlfactorial(::UInt32)
\end{lstlisting}

\hypertarget{summary-key-ingredients-of-performance}{%
\subsection{Summary: Key ingredients of
performance}\label{summary-key-ingredients-of-performance}}

\begin{itemize}
\tightlist
\item
  \textbf{JIT} compilation: compile the code with \textbf{LLVM} compiler
  framework when a method is called for the first time;
\item
  \textbf{Multiple dispatch}: invoke the correct method instance
  according to the type of multiple arguments;
\end{itemize}

\hypertarget{type-and-multiple-dispatch}{%
\section{Type and Multiple-dispatch}\label{type-and-multiple-dispatch}}

\hypertarget{julia-types}{%
\subsection{Julia Types}\label{julia-types}}

Julia has rich type system, which is not limited to the
\textbf{primitive types} that supported by the hardware. The type system
is the key to the \textbf{multiple dispatch} feature of Julia.

As an example, let us consider the type for complex numbers.

\begin{lstlisting}
Complex{Float64}
\end{lstlisting}

where \passthrough{\lstinline!Float64!} is the \textbf{type parameter}
of \passthrough{\lstinline!Complex!}. Type parameters are a part of a
type, without which the type is not fully specified. A fully specified
type is called a \textbf{concrete type}, which has a fixed memory layout
and can be instantiated in memory. For example, the
\passthrough{\lstinline!Complex\{Float64\}!} consists of two fields of
type \passthrough{\lstinline!Float64!}, which are the real and imaginary
parts of the complex number.

\begin{lstlisting}
julia> fieldnames(Complex{Float64})
(:re, :im)

julia> fieldtypes(Complex{Float64})
(Float64, Float64)
\end{lstlisting}

Extending the example, we can define the type for a matrix of complex
numbers.

\begin{lstlisting}
Array{Complex{Float64}, 2}
\end{lstlisting}

\passthrough{\lstinline!Array!} type has two type parameters, the first
one is the \textbf{element type} and the second one is the
\textbf{dimension} of the array.

One can get the type of value with \passthrough{\lstinline!typeof!}
function.

\begin{lstlisting}
julia> typeof(1+2im)

julia> typeof(randn(Complex{Float64}, 2, 2))
\end{lstlisting}

Then, what the type of a type?

\begin{lstlisting}
julia> typeof(Complex{Float64})
DataType
\end{lstlisting}

\hypertarget{example-define-you-first-type}{%
\subsection{Example: define you first
type}\label{example-define-you-first-type}}

We first define of an abstract type
\passthrough{\lstinline!AbstractAnimal!} with the keyword
\passthrough{\lstinline!abstract type!}:

\begin{lstlisting}
julia> abstract type AbstractAnimal{L} end
\end{lstlisting}

where the type parameter \passthrough{\lstinline!L!} stands for the
number of legs. Defining the number of legs as a type parameter or a
field of a concrete type is a design choice. Providing more information
in the type system can help the compiler to optimize the code, but it
can also make the compiler generate more code.

Abstract types can have subtypes. In the following we define a concrete
subtype type \passthrough{\lstinline!Dog!} with 4 legs, which is a
subtype of \passthrough{\lstinline!AbstractAnimal\{4\}!}.

\begin{lstlisting}
julia> struct Dog <: AbstractAnimal{4}
    color::String
end
\end{lstlisting}

where \passthrough{\lstinline!<:!} is the symbol for sybtyping,
\passthrough{\lstinline!A <: B!} means A is a subtype of B. Concrete
types can have fields, which are the data members of the type. However,
they can not have subtypes.

Similarly, we define a \passthrough{\lstinline!Cat!} with 4 legs, a
\passthrough{\lstinline!Cock!} with 2 legs and a
\passthrough{\lstinline!Human!} with 2 legs.

\begin{lstlisting}
julia> struct Cat <: AbstractAnimal{4}
    color::String
end

julia> struct Cock <: AbstractAnimal{2}
    gender::Bool
end

julia> struct Human{FT <: Real} <: AbstractAnimal{2}
    height::FT
    function Human(height::T) where T <: Real
        if height <= 0 || height > 300
            error("The tall of a Human being must be in range 0~300, got $(height)")
        end
        return new{T}(height)
    end
end
\end{lstlisting}

Here, the \passthrough{\lstinline!Human!} type has its own constructor.
The \passthrough{\lstinline!new!} function is the default constructor.

We can define a \textbf{fall back method}
\passthrough{\lstinline!fight!} on the abstract type
\passthrough{\lstinline!AbstractAnimal!}

\begin{lstlisting}
julia> fight(a::AbstractAnimal, b::AbstractAnimal) = "draw"
\end{lstlisting}

where \passthrough{\lstinline!::!} is a type assertion. This function
will be invoked if two subtypes of
\passthrough{\lstinline!AbstractAnimal!} are fed into the function
\passthrough{\lstinline!fight!} and no more \textbf{explicit} methods
are defined.

We can define many more explicit methods with the same name.

\begin{lstlisting}
julia> fight(dog::Dog, cat::Cat) = "win"
fight (generic function with 2 methods)

julia> fight(hum::Human, a::AbstractAnimal) = "win"
fight (generic function with 3 methods)

julia> fight(hum::Human, a::Union{Dog, Cat}) = "loss"
fight (generic function with 4 methods)

julia> fight(hum::AbstractAnimal, a::Human) = "loss"
fight (generic function with 5 methods)
\end{lstlisting}

where \passthrough{\lstinline!Union\{Dog, Cat\}!} is a \textbf{union
type}. It is a type that can be either \passthrough{\lstinline!Dog!} or
\passthrough{\lstinline!Cat!}. \passthrough{\lstinline!Union!} types are
not concrete since they do not have a fixed memory layout, meanwhile,
they can not be subtyped! Here, we defined 5 methods for the function
\passthrough{\lstinline!fight!}. However, defining too many methods for
the same function can be dangerous. You need to be careful about the
ambiguity error!

\begin{lstlisting}
julia> fight(Human(170), Human(180))
ERROR: MethodError: fight(::Human{Int64}, ::Human{Int64}) is ambiguous.

Candidates:
  fight(hum::AbstractAnimal, a::Human)
    @ Main REPL[37]:1
  fight(hum::Human, a::AbstractAnimal)
    @ Main REPL[35]:1

Possible fix, define
  fight(::Human, ::Human)

Stacktrace:
 [1] top-level scope
   @ REPL[38]:1
\end{lstlisting}

It makes sense because we claim \passthrough{\lstinline!Human!} wins any
other animals, but we also claim any animal losses to
\passthrough{\lstinline!Human!}. When it comes to two
\passthrough{\lstinline!Human!}s, the two functions are equally valid.
To resolve the ambiguity, we can define a new method for the function
\passthrough{\lstinline!fight!} as follows.

\begin{lstlisting}
julia> fight(hum::Human{T}, hum2::Human{T}) where T<:Real = hum.height > hum2.height ? "win" : "loss"
\end{lstlisting}

Now, we can test the function \passthrough{\lstinline!fight!} with
different combinations of animals.

\begin{lstlisting}
julia> fight(Cock(true), Cat("red"))
"draw"

julia> fight(Dog("blue"), Cat("white"))
"win"

julia> fight(Human(180), Cat("white"))
"loss"

julia> fight(Human(170), Human(180))
"loss"
\end{lstlisting}

Quiz: How many method instances are generated for fight so far?

\begin{lstlisting}
julia> methodinstances(fight)
\end{lstlisting}

\hypertarget{julia-number-system}{%
\subsection{Julia number system}\label{julia-number-system}}

The type tree rooted on \passthrough{\lstinline!Number!} looks like:

\begin{lstlisting}
Number
├─ Base.MultiplicativeInverses.MultiplicativeInverse{T}
│  ├─ Base.MultiplicativeInverses.SignedMultiplicativeInverse{T<:Signed}
│  └─ Base.MultiplicativeInverses.UnsignedMultiplicativeInverse{T<:Unsigned}
├─ Complex{T<:Real}
├─ Real
│  ├─ AbstractFloat
│  │  ├─ BigFloat
│  │  ├─ Float16
│  │  ├─ Float32
│  │  └─ Float64
│  ├─ AbstractIrrational
...
\end{lstlisting}

The Julia type system is a tree, and \passthrough{\lstinline!Any!} is
the root of type tree, i.e.~it is a super type of any other type. The
\passthrough{\lstinline!Number!} type is the root type of julia number
system, which is also a subtype of \passthrough{\lstinline!Any!}.

\begin{lstlisting}
julia> Number <: Any
\end{lstlisting}

There are utilities to analyze the type system:

\begin{lstlisting}
julia> subtypes(Number)
3-element Vector{Any}:
 Base.MultiplicativeInverses.MultiplicativeInverse
 Complex
 Real

julia> supertype(Float64)
AbstractFloat

julia> AbstractFloat <: Real
true
\end{lstlisting}

The leaf nodes of the type tree are called \textbf{concrete types}. They
are the types that can be instantiated in memory. Among the concrete
types, there are \textbf{primitive types} and \textbf{composite types}.
Primitive types are built into the language, such as
\passthrough{\lstinline!Int64!}, \passthrough{\lstinline!Float64!},
\passthrough{\lstinline!Bool!}, and \passthrough{\lstinline!Char!},
while are built on top of primitive types, such as
\passthrough{\lstinline!Dict!}, \passthrough{\lstinline!Complex!} and
the user-defined types.

\textbf{The list of primitive types}

\begin{lstlisting}[language=bash]
primitive type Float16 <: AbstractFloat 16 end
primitive type Float32 <: AbstractFloat 32 end
primitive type Float64 <: AbstractFloat 64 end

primitive type Bool <: Integer 8 end
primitive type Char <: AbstractChar 32 end

primitive type Int8    <: Signed   8 end
primitive type UInt8   <: Unsigned 8 end
primitive type Int16   <: Signed   16 end
primitive type UInt16  <: Unsigned 16 end
primitive type Int32   <: Signed   32 end
primitive type UInt32  <: Unsigned 32 end
primitive type Int64   <: Signed   64 end
primitive type UInt64  <: Unsigned 64 end
primitive type Int128  <: Signed   128 end
primitive type UInt128 <: Unsigned 128 end
\end{lstlisting}

\hypertarget{extending-the-number-system}{%
\subsection{Extending the number
system}\label{extending-the-number-system}}

Extending the number system in Julia is much easier than in
object-oriented languages like Python. In the following example, we show
how to implement addition operation of a user defined class in Python.

\begin{lstlisting}[language=Python]
class X:
  def __init__(self, num):
    self.num = num

  def __add__(self, other_obj):
    return X(self.num+other_obj.num)

  def __radd__(self, other_obj):
    return X(other_obj.num + self.num)

  def __str__(self):
    return "X = " + str(self.num)

class Y:
  def __init__(self, num):
    self.num = num

  def __radd__(self, other_obj):
    return Y(self.num+other_obj.num)

  def __str__(self):
    return "Y = " + str(self.num)

print(X(3) + Y(5))

print(Y(3) + X(5))
\end{lstlisting}

Here, we implemented the addition operation of two classes
\passthrough{\lstinline!X!} and \passthrough{\lstinline!Y!}. The
\passthrough{\lstinline!\_\_add\_\_!} method is called when the
\passthrough{\lstinline!+!} operator is used with the object on the
left-hand side, while the \passthrough{\lstinline!\_\_radd\_\_!} method
is called when the object is on the right-hand side. The output is as
follows:

\begin{lstlisting}
X = 8
X = 8
\end{lstlisting}

It turns out the \passthrough{\lstinline!\_\_radd\_\_!} method of
\passthrough{\lstinline!Y!} is not called at all. This is because the
\passthrough{\lstinline!\_\_radd\_\_!} method is only called when the
object on the left-hand side does not have the
\passthrough{\lstinline!\_\_add\_\_!} method by some artifical rules.

Implement addition in Julian style is much easier. We can define the
addition operation of two types \passthrough{\lstinline!X!} and
\passthrough{\lstinline!Y!} as follows.

\begin{lstlisting}
julia> struct X{T} <: Number
    num::T
end

julia> struct Y{T} <: Number
    num::T
end

julia> Base.:(+)(a::X, b::Y) = X(a.num + b.num);

julia> Base.:(+)(a::Y, b::X) = X(a.num + b.num);

julia> Base.:(+)(a::X, b::X) = X(a.num + b.num);

julia> Base.:(+)(a::Y, b::Y) = Y(a.num + b.num);
\end{lstlisting}

Multiple dispatch seems to be more expressive than object-oriented
programming.

Now, supposed you want to extend this method to a new type
\passthrough{\lstinline!Z!}. In python, he needs to define a new class
\passthrough{\lstinline!Z!} as follows.

\begin{lstlisting}[language=Python]
class Z:
  def __init__(self, num):
    self.num = num

  def __add__(self, other_obj):
    return Z(self.num+other_obj.num)

  def __radd__(self, other_obj):
    return Z(other_obj.num + self.num)

  def __str__(self):
    return "Z = " + str(self.num)

print(X(3) + Z(5))

print(Z(3) + X(5))
\end{lstlisting}

The output is as follows:

\begin{lstlisting}
X = 8
Z = 8
\end{lstlisting}

No matter how hard you try, you can not make the
\passthrough{\lstinline!\_\_add\_\_!} method of
\passthrough{\lstinline!Z!} to be called when the object is on the
left-hand side. In Julia, this is not a problem at all. We can define
the addition operation of \passthrough{\lstinline!Z!} as follows.

\begin{lstlisting}
julia> struct Z{T} <: Number
    num::T
end

julia> Base.:(+)(a::X, b::Z) = Z(a.num + b.num);

julia> Base.:(+)(a::Z, b::X) = Z(a.num + b.num);

julia> Base.:(+)(a::Y, b::Z) = Z(a.num + b.num);

julia> Base.:(+)(a::Z, b::Y) = Z(a.num + b.num);

julia> Base.:(+)(a::Z, b::Z) = Z(a.num + b.num);

julia> X(3) + Y(5)
X{Int64}(8)

julia> Y(3) + X(5)
X{Int64}(8)

julia> X(3) + Z(5)
Z{Int64}(8)

julia> Z(3) + Y(5)
Z{Int64}(8)
\end{lstlisting}

There is a deeper reason why multiple dispatch is more expressive than
object-oriented programming. The Julia function space is exponentially
large! If a function \(f\) has \(k\) parameters, and the module has
\(t\) types, there can be \(t^k\) methods for the function \(f\).

\begin{lstlisting}
f(x::T1, y::T2, z::T3...)
\end{lstlisting}

However, in an object-oriented language like Python, the function space
is only linear to the number of classes.

\begin{lstlisting}[language=Python]
class T1:
    def f(self, y, z, ...):
        self.num = num
\end{lstlisting}

\hypertarget{example-fibonacci-number}{%
\subsection{Example: Fibonacci number}\label{example-fibonacci-number}}

The Fibonacci number is defined as follows.

\begin{lstlisting}
julia> fib(x::Int) = x <= 2 ? 1 : fib(x-1) + fib(x-2)
fib (generic function with 1 methods)

julia> addup(x::Int, y::Int) = x + y
addup (generic function with 1 methods)

julia> @btime fib(40)
  278.066 ms (0 allocations: 0 bytes)
102334155
\end{lstlisting}

Oops, it is really slow. There is definitely a better way to calculate
the Fibonacci number, but let us stick to the current implementation for
now.

If you know the Julia type system, you can implement the Fibonacci
number in a zero cost way. The trick is to use the type system to
calculate the Fibonacci number at compile time. There is a type
\passthrough{\lstinline!Val!} defined in the
\passthrough{\lstinline!Base!} module, which is just a type with a type
parameter. The type parameter can be a number:

\begin{lstlisting}
julia> Val(3.0)
Val{3.0}()
\end{lstlisting}

We can define the addition operation of \passthrough{\lstinline!Val!} as
the addition of the type parameters.

\begin{lstlisting}
julia> addup(::Val{x}, ::Val{y}) where {x, y} = Val(x + y)
addup (generic function with 2 methods)

julia> addup(Val(5), Val(7))
Val{12}()
\end{lstlisting}

Finally, we can define the Fibonacci number in a zero cost way.

\begin{lstlisting}
julia> fib(::Val{x}) where x = x <= 2 ? Val(1) : addup(fib(Val(x-1)), fib(Val(x-2)))
fib (generic function with 2 methods)


julia> @btime fib(Val(40))
  0.792 ns (0 allocations: 0 bytes)
Val{102334155}()
\end{lstlisting}

Wow, it is really fast! However, this trick is not recommended. It is
not a good practice to abuse the type system. You simply transfer the
run-time to compile time, which violates the
\href{https://docs.julialang.org/en/v1/manual/performance-tips/}{Performance
Tips}. On the other hand, we find the compiling time of the function
\passthrough{\lstinline!fib!} is much shorter than the run-time. This is
because the function \passthrough{\lstinline!fib!} is a
\textbf{recursive function}. The compiler can not optimize the recursive
function very well.

\hypertarget{summary}{%
\subsection{Summary}\label{summary}}

\begin{itemize}
\tightlist
\item
  \emph{Multiple dispatch} is a feature of some programming languages in
  which a function or method can be dynamically dispatched based on the
  \textbf{run-time} type.
\item
  Julia's mutiple dispatch provides exponential abstraction power
  comparing with an object-oriented language.
\item
  By carefully designed type system, we can program in an exponentially
  large function space.
\end{itemize}

\hypertarget{tuple-array-and-broadcasting}{%
\section{Tuple, Array and
broadcasting}\label{tuple-array-and-broadcasting}}

\textbf{Tuple has fixed memory layout, but array does not.}

\begin{lstlisting}[language=Julia]
tp = (1, 2.0, 'c')
\end{lstlisting}

\begin{lstlisting}[language=Output]
(1, 2.0, 'c')
\end{lstlisting}

\begin{lstlisting}[language=Julia]
typeof(tp)
\end{lstlisting}

\begin{lstlisting}[language=Output]
Tuple{Int64, Float64, Char}
\end{lstlisting}

\begin{lstlisting}[language=Julia]
isbitstype(typeof(tp))
\end{lstlisting}

true

\begin{lstlisting}[language=Julia]
arr = [1, 2.0, 'c']
\end{lstlisting}

1

2.0

\begin{lstlisting}[language=Output]
'c': ASCII/Unicode U+0063 (category Ll: Letter, lowercase)
\end{lstlisting}

\begin{lstlisting}[language=Julia]
typeof(arr)
\end{lstlisting}

\begin{lstlisting}[language=Output]
Vector{Any} (alias for Array{Any, 1})
\end{lstlisting}

\begin{lstlisting}[language=Julia]
isbitstype(typeof(arr))
\end{lstlisting}

false

\textbf{Boardcasting}

\begin{lstlisting}[language=Julia]
x = 0:0.1:π
\end{lstlisting}

\begin{lstlisting}[language=Julia]
y = sin.(x)
\end{lstlisting}

\begin{lstlisting}[language=Julia]
using Plots
\end{lstlisting}

\begin{lstlisting}[language=Julia]
plot(x, y; label="sin")
\end{lstlisting}

\begin{lstlisting}[language=Julia]
mesh = (1:100)'
\end{lstlisting}

\textbf{Broadcasting over non-concrete element types may be type
unstable.}

\begin{lstlisting}[language=Julia]
eltype(arr)
\end{lstlisting}

\begin{lstlisting}[language=Output]
Any
\end{lstlisting}

\begin{lstlisting}[language=Julia]
arr .+ 1
\end{lstlisting}

2

3.0

\begin{lstlisting}[language=Output]
'd': ASCII/Unicode U+0064 (category Ll: Letter, lowercase)
\end{lstlisting}

\begin{lstlisting}[language=Julia]
eltype(tp)
\end{lstlisting}

\begin{lstlisting}[language=Output]
Any
\end{lstlisting}

\hypertarget{julia-package-development}{%
\subsection{Julia package development}\label{julia-package-development}}

\begin{lstlisting}[language=Julia]
using TropicalNumbers
\end{lstlisting}

\begin{lstlisting}[language=Output]
nothing
\end{lstlisting}

The file structure of a package

\begin{lstlisting}[language=Julia]
project_folder = dirname(dirname(pathof(TropicalNumbers)))
\end{lstlisting}

/home/runner/.julia/packages/TropicalNumbers/kRhOl

\textbf{Unit Test}

\begin{lstlisting}[language=Julia]
using Test
\end{lstlisting}

\begin{lstlisting}[language=Output]
nothing
\end{lstlisting}

\begin{lstlisting}[language=Julia]
@test Tropical(3.0) + Tropical(2.0) == Tropical(3.0)
\end{lstlisting}

\begin{lstlisting}[language=Output]
Test Passed
\end{lstlisting}

\begin{lstlisting}[language=Julia]
@test_throws BoundsError [1,2][3]
\end{lstlisting}

\begin{lstlisting}[language=Output]
Test Passed
      Thrown: BoundsError
\end{lstlisting}

\begin{lstlisting}[language=Julia]
@test_broken 3 == 2
\end{lstlisting}

\begin{lstlisting}[language=Output]
Test Broken
  Expression: 3 == 2
\end{lstlisting}

\begin{lstlisting}[language=Julia]
@testset "Tropical Number addition" begin
    @test Tropical(3.0) + Tropical(2.0) == Tropical(3.0)
    @test_throws BoundsError [1][2]
    @test_broken 3 == 2
end
\end{lstlisting}

\begin{lstlisting}[language=Output]
Test.DefaultTestSet("Tropical Number addition", Any[Test Broken
  Expression: 3 == 2], 2, false, false, true, 1.707316862782449e9, 1.707316862807562e9, false, "none")
\end{lstlisting}

\hypertarget{case-study-create-a-package-like-happymolecules}{%
\subsection{Case study: Create a package like
HappyMolecules}\label{case-study-create-a-package-like-happymolecules}}

With \passthrough{\lstinline!PkgTemplates!}.

\url{https://github.com/CodingThrust/HappyMolecules.jl}

\begin{lstlisting}
julia> isbitstype(Complex{Float64})

julia> sizeof(Complex{Float32})

julia> sizeof(Complex{Float64})
\end{lstlisting}

But \passthrough{\lstinline!Complex\{BigFloat\}!} is not

\begin{lstlisting}
julia> sizeof(Complex{BigFloat})

julia> isbitstype(Complex{BigFloat})
\end{lstlisting}

The size of \passthrough{\lstinline!Complex\{BigFloat\}!} is not true!
It returns the pointer size!

\hypertarget{how-to-measure-the-performance-of-your-cpu}{%
\subsection{How to measure the performance of your
CPU?}\label{how-to-measure-the-performance-of-your-cpu}}

The performance of a CPU is measured by the number of \textbf{floating
point operations per second} (FLOPS) it can perform. The floating point
operations include addition, subtraction, multiplication and division.
The FLOPS can be related to multiple factors, such as the clock
frequency, the number of cores, the number of instructions per cycle,
and the number of floating point units. A simple way to measure the
FLOPS is to benchmarking the speed of matrix multiplication.

\begin{lstlisting}
julia> using BenchmarkTools

julia> A, B = rand(1000, 1000), rand(1000, 1000);

julia> @btime $A * $B;
  12.122 ms (2 allocations: 7.63 MiB)
\end{lstlisting}

The number of FLOPS in a \(n\times n\times n\) matrix multiplication is
\(2n^3\). The FLOPS can be calculated as:
\(2 \times 1000^3 / (12.122 \times 10^{-3}) = 165~{\rm GFLOPS}\).

\hypertarget{case-study-vector-element-type-and-speed}{%
\subsection{Case study: Vector element type and
speed}\label{case-study-vector-element-type-and-speed}}

Any type vector is flexible. You can add any element into it.

\begin{lstlisting}
vany = Any[]  # same as vany = []

typeof(vany)

push!(vany, "a")

push!(vany, 1)
\end{lstlisting}

Fixed typed vector is more restrictive.

\begin{lstlisting}
vfloat64 = Float64[]

vfloat64 |> typeof

push!(vfloat64, "a")
\end{lstlisting}

Do not abuse the type system. e.g.~a ``zero'' cost implementation

\begin{lstlisting}
Val(3.0) # just a type

f(::Val{1}) = Val(1)

f(::Val{2}) = Val(1)
\end{lstlisting}

It violates the
\href{https://docs.julialang.org/en/v1/manual/performance-tips/}{Performance
Tips}, since it transfers the run-time to compile time.

\begin{lstlisting}
let biganyv = collect(Any, 1:2:20000)
    @benchmark for i=1:length($biganyv)
        $biganyv[i] += 1
    end
end
\end{lstlisting}

\begin{lstlisting}
let bigfloatv = collect(Float64, 1:2:20000)
    @benchmark for i=1:length($bigfloatv)
        $bigfloatv[i] += 1
    end
end
\end{lstlisting}

\begin{lstlisting}
fib(x::Int) = x <= 2 ? 1 : fib(x-1) + fib(x-2)

@benchmark fib(20)
\end{lstlisting}

\begin{lstlisting}
addup(::Val{x}, ::Val{y}) where {x, y} = Val(x + y)
\end{lstlisting}

\begin{lstlisting}
f(::Val{x}) where x = addup(f(Val(x-1)), f(Val(x-2)))
\end{lstlisting}

\begin{lstlisting}
@benchmark f(Val(20)) end
\end{lstlisting}

\hypertarget{sec:publishing-package}{%
\section{Publishing a Package}\label{sec:publishing-package}}

Now that you have an amazing package, it's time to make it available to
the public. Before that, there is one final task to be done which is to
choose a license.

\begin{itemize}
\item
  GNU's Not Unix! (GNU) (1983 by Richard Stallman)

  Its goal is to give computer users freedom and control in their use of
  their computers
  and~\href{https://en.wikipedia.org/wiki/Computer_hardware}{computing
  devices}~by collaboratively developing and publishing software that
  gives everyone the rights to freely run the software, copy and
  distribute it, study it, and modify it. GNU software grants these
  rights in
  its~\href{https://en.wikipedia.org/wiki/GNU_General_Public_License}{license}.
\item
  The problem of GPL Lisense: The GPL and licenses modeled on it impose
  the restriction that source code must be distributed or made available
  for all works that are derivatives of the GNU copyrighted code.

  Case study:
  \href{https://www.notion.so/Wiki-53dd9dafd57b40f6b253d6605667a472}{Free
  Software fundation v.s. Cisco Systems}

  Modern Licenses are:
  \href{https://en.wikipedia.org/wiki/MIT_License}{MIT} and
  \href{https://en.wikipedia.org/wiki/Apache_License}{Apache}.
\end{itemize}

\hypertarget{appendix}{%
\chapter*{Appendix}\label{appendix}}
\addcontentsline{toc}{chapter}{Appendix}

This is the appendix.

\hypertarget{references}{%
\chapter{References}\label{references}}

\backmatter

\end{document}
